% Fuzzy Set Theory Background and Methods

\noindent
% Briefly recall the basic definitions of fuzzy sets and fuzzy logic.
% A fuzzy set \(A\) on a universe of discourse \(U\) is a pair \((U,\mu_A)\)
% where the membership function \(\mu_A:U\to[0,1]\) assigns to each element
% a degree of membership \citep{WikipediaFuzzySet}.  Unlike crisp sets,
% membership grades allow gradual transitions between full inclusion (1) and
% exclusion (0).  Fuzzy logic, introduced by \citet{Zadeh1965}, extends
% classical Boolean logic to handle partial truth values and supports
% approximate reasoning.

% \subsection*{Feature extraction}
% We reuse the same features extracted for the statistical analysis
% (Section~\ref{sec:methods} of the statistics paper).  These include
% measures of sentence length, lexical diversity, character entropy,
% function word proportion, first person pronoun ratio, bigram repetition
% and readability.  Each feature provides evidence for the degree to which
% a text is more likely to be human or machine–generated.

% \subsection*{Membership functions}
% For each feature \(x\) we construct three triangular membership
% functions representing the linguistic concepts ``low'', ``medium'' and
% ``high''.  The vertices of these triangles are determined from the
% 33rd percentile \(q_{0.33}\), median \(q_{0.50}\) and 66th percentile
% \(q_{0.66}\) of the feature distribution across all samples (training
% data).  Let \(x_{\min}\) and \(x_{\max}\) denote the minimum and
% maximum observed values.  Then the ``low'' membership function has
% vertices \((x_{\min}, x_{\min}, q_{0.33})\), the ``medium'' function
% \((q_{0.33}, q_{0.50}, q_{0.66})\) and the ``high'' function
% \((q_{0.33}, q_{0.66}, x_{\max})\).  The membership degree of a value
% \(x\) is computed using the standard triangular function.

% \subsection*{Orientation and rules}
% To determine whether a feature indicates humanness or machine–
% generatedness, we examine group medians: if the median of the human
% class exceeds that of the LLM class, the feature is considered
% ``direct'' and high values support the human hypothesis; otherwise
% orientation is ``inverse'' and low values support the human hypothesis.
% Each feature thus casts a vote for the human or LLM class based on its
% membership degree.  Medium membership contributes equally to both classes.

% \subsection*{Inference and defuzzification}
% For a given sample, membership degrees for all features are computed and
% aggregated by averaging to obtain overall scores for the human and LLM
% hypotheses.  The probabilities are normalised to sum to 1 and the
% predicted class is the one with the higher degree.  This simple
% aggregation corresponds to a Takagi–Sugeno zero‑order model with
% uniformly weighted rules \citep{Zimmermann2001}.  More sophisticated
% aggregation operators (e.g. weighted averages, fuzzy integrals) could
% be explored in future work.

% \subsection*{Evaluation}
% The fuzzy classifier is evaluated using the same cross–validation
% scheme as the statistical models (leave–one–topic–out or
% stratified K–fold).  Performance is measured by accuracy, ROC curves
% and precision–recall curves, allowing a direct comparison with the
% classical classifiers.  The transparency of the fuzzy system makes it
% straightforward to analyse which features contribute most to the
% decision.

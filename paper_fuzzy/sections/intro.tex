% Introduction for fuzzy logic paper

\noindent
% Provide an overview of fuzzy set theory and its motivation in modelling
% vagueness and uncertainty.  Fuzzy sets allow elements to belong to a
% set with degrees of membership rather than crisp yes/no assignments
% \citep{Zadeh1965}.  A fuzzy set is defined by a universe of discourse
% and a membership function mapping each element to the interval [0,1]
% \citep{WikipediaFuzzySet}.  This generalisation of classical set
% theory permits gradual assessment of membership and has been applied in
% linguistics, decision making and control systems.

% In the context of natural language generation, distinguishing human
% texts from machine–generated texts often requires dealing with
% uncertainties: certain stylometric features may indicate a text is
% human–like but not conclusively so.  Fuzzy logic offers a framework
% for approximate reasoning \citep{Zimmermann2001} that can complement
% crisp statistical models.  By assigning degrees of belief to the
% hypotheses "human" and "LLM", and aggregating evidence from
% multiple features, fuzzy inference systems can provide interpretable
% classifications.

% State the purpose of this paper: to apply fuzzy set theory and fuzzy
% inference to the same stylometric feature set used in the statistical
% analysis, construct membership functions from the data, derive simple
% inference rules automatically and evaluate the performance of this
% approach.  Highlight that by reusing the extracted features we
% minimise additional work while exploring a complementary methodology.

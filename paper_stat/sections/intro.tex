% Introduction

\noindent
A emergência de modelos de linguagem de grande porte (LLMs) criou preocupações quanto à detecção de conteúdo gerado automaticamente. A detecção de autoria computacional tem raízes históricas sólidas, iniciando com o trabalho seminal de Mosteller e Wallace~\cite{mosteller1964} sobre os artigos Federalistas e posteriormente formalizada por Burrows~\cite{burrows2002} com a medida Delta para diferenciação estilística. Trabalhos recentes demonstram que essas técnicas estilométricas clássicas permanecem eficazes para distinguir textos autorais de textos gerados por LLMs~\cite{stamatatos2009,huang2024}.

Estudos em múltiplos idiomas confirmam a viabilidade da abordagem estilométrica: Herbold et al.~\cite{stylometric_llm_detection} reportaram 81--98\% de acurácia usando 31 características e floresta aleatória; Zaitsu e Jin~\cite{zaitsu2023} alcançaram 100\% de precisão em textos japoneses; Przystalski et al.~\cite{przystalski2025} demonstraram que estilometria reconhece LLMs mesmo em pequenas amostras (0,87--0,98 de acurácia); e Berriche e Larabi-Marie-Sainte~\cite{berriche2024} atingiram 100\% usando 33 características estilométricas com \textit{XGBoost}. Esses resultados evidenciam que características como comprimento médio de frases, relação tipo-token, entropia de caracteres~\cite{shannon1948}, proporção de palavras funcionais~\cite{stamatatos2009} e burstiness (variação estrutural)~\cite{gptzero2023,chakraborty2023ct2} contêm sinais fortes sobre a origem do texto.

Este estudo contribui para a literatura de detecção de LLMs ao fornecer uma primeira análise estilométrica para detecção de textos gerados por LLMs em português do Brasil. Não foi encontrado aplicação de análise estilométrica a textos de LLM em português. Utilizou-se um conjunto de dados balanceado com mais de 1,2 milhões de amostras de múltiplas fontes (BrWaC~\cite{brwac}, ShareGPT-Portuguese~\cite{sharegpt_portuguese}, Canarim~\cite{canarim}).\footnote{Desde a compilação deste corpus, novos recursos em português surgiram, incluindo GigaVerbo com 200B tokens~\cite{correa2024} e PTT5-v2~\cite{piau2024}, que podem beneficiar trabalhos futuros.}

\subsection{Mineração de Texto}

A mineração de texto é o processo de extração de informação relevante e conhecimento a partir de dados textuais não estruturados \cite{feldman2007}. Diferentemente da análise de dados tabulares tradicionais, a mineração de texto requer a transformação de documentos em representações numéricas que possibilitem a aplicação de métodos estatísticos.

O processo de mineração de texto compreende quatro etapas fundamentais:

\begin{enumerate}
    \item \textbf{Coleta de dados}: Aquisição de documentos textuais de fontes diversas, garantindo representatividade da população de interesse.

    \item \textbf{Pré-processamento}: Limpeza e normalização dos textos, incluindo remoção de caracteres especiais, normalização de espaços em branco, e conversão para codificação uniforme (UTF-8).

    \item \textbf{Extração de características}: Transformação dos documentos em vetores de variáveis quantitativas mensuráveis. Esta etapa é crucial pois define as variáveis que serão analisadas estatisticamente.

    \item \textbf{Análise estatística}: Aplicação de métodos estatísticos descritivos e inferenciais sobre as variáveis extraídas para identificar padrões, diferenças entre grupos, e construir modelos preditivos.
\end{enumerate}

No contexto deste trabalho, a mineração de texto serve como ponte entre documentos textuais brutos e a análise estatística formal. As características extraídas (descritas na Seção \ref{sec:features}) são variáveis quantitativas mensuradas em escalas de razão ou intervalo, permitindo a aplicação de métodos estatísticos paramétricos e não paramétricos.

\subsection{Estilometria e Análise de Autoria}

A estilometria é o estudo quantitativo do estilo linguístico através da medição de características objetivas dos textos \cite{stamatatos2009}. Fundamenta-se no princípio de que autores possuem padrões linguísticos inconscientes e consistentes que podem ser identificados estatisticamente.

\subsubsection{Fundamentos da Análise Estilométrica}

A análise estilométrica baseia-se em três premissas fundamentais:

\begin{enumerate}
    \item \textbf{Consistência autoral}: Autores humanos mantêm padrões estilísticos relativamente estáveis ao longo de diferentes textos e tópicos.

    \item \textbf{Variabilidade inter-autoral}: As diferenças estilísticas entre autores distintos são maiores que as variações intra-autorais.

    \item \textbf{Mensurabilidade}: Características estilísticas podem ser quantificadas através de variáveis mensuráveis objetivamente.
\end{enumerate}

O trabalho seminal de \textcite{mosteller1964} sobre os \textit{Federalist Papers} demonstrou que métodos estatísticos rigorosos podem atribuir autoria com alta confiança. A abordagem foi posteriormente formalizada por \textcite{burrows2002} com a medida Delta, que utiliza distâncias estatísticas entre perfis estilométricos.

\subsubsection{Características Estilométricas}

As variáveis estilométricas utilizadas em análise de autoria podem ser categorizadas conforme suas escalas de medida:

\textbf{Variáveis em escala de razão} (possuem zero absoluto e razões interpretáveis):
\begin{itemize}
    \item Comprimento médio de frase (palavras por frase)
    \item Frequência de uso de pontuação específica (por 1000 palavras)
    \item Riqueza lexical (razão tipo-token)
    \item Proporções de classes gramaticais (substantivos, verbos, etc.)
\end{itemize}

\textbf{Variáveis em escala de intervalo} (diferenças interpretáveis, mas sem zero absoluto):
\begin{itemize}
    \item Entropia de distribuição de caracteres \cite{shannon1948}
    \item Coeficiente de variação do comprimento de frase (\textit{burstiness} normalizado) \cite{gptzero2023,chakraborty2023ct2}
\end{itemize}

A distinção entre escalas de medida é fundamental porque determina quais métodos estatísticos são aplicáveis. Variáveis em escala de razão permitem operações aritméticas completas e cálculo de medidas como média geométrica e coeficiente de variação. Variáveis em escala de intervalo permitem cálculo de médias e desvios padrão, mas não razões.

\subsubsection{Detecção de Textos Gerados por LLMs}

Estudos recentes demonstram que técnicas estilométricas clássicas permanecem eficazes para detectar textos gerados por modelos de linguagem de grande porte \cite{stylometric_llm_detection,stamatatos2009}. \textcite{stylometric_llm_detection} reportaram acurácia superior a 99\% utilizando características estilométricas simples em amostras curtas (100-200 palavras).

Trabalhos específicos para o português incluem \textcite{berriche2024}, que demonstraram a eficácia de medidas de entropia de caracteres e proporção de palavras funcionais. O presente trabalho estende essa linha de pesquisa aplicando métodos estatísticos multivariados a um conjunto abrangente de características estilométricas em português do Brasil.

\subsection{Justificativa para Múltiplos Métodos Estatísticos}

Este trabalho aplica três métodos multivariados complementares (PCA, LDA, Regressão Logística) por razões metodológicas distintas, não redundantes:

\begin{enumerate}
    \item \textbf{PCA - Análise Exploratória}: Método não supervisionado para visualização de estrutura natural dos dados e identificação de padrões sem conhecimento prévio das classes. Responde: \textit{``As variáveis se agrupam naturalmente por categoria (humano/LLM) sem supervisão?''}

    \item \textbf{LDA - Discriminação Ótima}: Método supervisionado que maximiza separação entre grupos conhecidos. Enquanto PCA maximiza variância total, LDA maximiza variância \textit{between-group} relativa à \textit{within-group}. Responde: \textit{``Qual combinação linear de variáveis melhor discrimina os grupos?''}

    \item \textbf{Regressão Logística - Modelagem Preditiva}: Método probabilístico que quantifica contribuição individual de cada variável e permite interpretação através de odds ratios. Responde: \textit{``Qual a probabilidade de um novo texto ser humano dado seu perfil estilométrico?''}
\end{enumerate}

\textbf{Complementaridade metodológica}:
\begin{itemize}
    \item PCA é \textbf{descritivo} (sem hipóteses)
    \item LDA é \textbf{discriminativo} (maximiza separação)
    \item Regressão Logística é \textbf{preditivo e inferencial} (estima probabilidades e testa significância)
\end{itemize}

Esta abordagem triangulada fortalece as conclusões: se os três métodos independentes convergem para as mesmas variáveis como importantes, aumenta a confiança na robustez dos achados.

% Methods

\noindent
% Describe the dataset (source, size, language) and how it was preprocessed.
% Outline the feature extraction procedure, referencing the metrics implemented
% in \texttt{src/features.py}: sentence statistics, lexical diversity, character
% entropy, function word ratio, pronoun ratio, bigram repetition and
% readability.  Specify which tools/libraries were used.

% Explain the statistical tests: Mann--Whitney U test for independent samples
% to compare feature distributions between human and LLM texts; Cliff's
% \(\delta\) as a measure of effect size; false discovery rate (FDR) control
% using the Benjamini--Hochberg procedure.

% Describe the multivariate models: PCA for dimensionality reduction and
% visualisation; linear discriminant analysis (LDA) and logistic regression
% for classification.  Note the cross--validation strategy (leave--one--topic--out
% if a topic column is present, otherwise stratified K--fold) and the metrics
% computed (ROC, AUC, precision--recall).  Clarify any hyperparameter
% choices.
